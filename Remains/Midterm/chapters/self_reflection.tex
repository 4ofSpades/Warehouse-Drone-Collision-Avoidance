\chapter{Self Reflection}
\lhead{\thechapter \space Self Reflection}
\label{ch:self_reflection}

This self reflection is divided into 3 parts: the specific activities/competences according to the domain description for IT bachelors, research skills, and professional behavior. Each part will include a short definition, a reflection on how this project is progressing with respect to that part, and a plan to improve on the current status.

\section{Domain Specific Competences}
According to the official regulations for graduation internships, it is required for an intern to display an equivalent to the N3-level amount of skill within the internship period. This roughly means that the selected activities should have a generally complex and unstructured nature, which should be fulfilled by the intern independently. The selected activities have to be 3 out of the following: manage, analyze, advise, design, implement. Furthermore, these activities have to be within at least 1 architectural layer (user interaction, business processes, infrastructure, software, hardware interfacing). The activities for this project are within the software layer, and are as follows:
\begin{itemize}
	\setlength\itemsep{-0.5em}
	\item \textbf{Advise:} Provide research-based advice on the most suitable approach for collision avoidance when keeping the requirements and wishes in consideration. 
	\item\textbf{Design:} Design an adjustable/scalable simulation environment for testing and/or training the solution.
	\item \textbf{Implement:} Implement the simulation and train (if applicable), optimize, and tailor the algorithm for collision avoidance. The algorithm should also be connected to the simulation.
\end{itemize}

\noindent
Starting off, the advice part is making good progress. Evidence is gathered based on comparisons, already existing papers/research, and own research/prototypes. Unfortunately, the design and implementation are not up to standard yet. As explained in section \ref{sec:collision_avoidance_status}, the current approach is unlikely to be completed within the remaining time scope, and while the simulation software is functional, does not fulfill the requirements for being a N3-level design activity. Having this happen was a known risk, yet it was still chosen based on the company's wishes and requirements and in order to satisfy the research part. Looking back the most time-costly mistake was to not make a prototype of the simulation to train on first. This was considered before starting, but based on research done the fact that the training environment needs to be very similar to the real one came out, which in combinations with the wishes of the company led to the current decision.
\\\\
As ultimately the most important task is to pass, the assignment will get some minor adjustments so that all competences can be adequately fulfilled. This will be done during the weeks of the midterm reports and presentations, so that both the feedback of the presentation as well as the wishes of the company can be incorporated into the new plan.

\section{Research Skills}
When assessing one's research skills, thing such as ability	to use state-of-the-art techniques	and	technologies, providing sound	research with proper research questions and unbiased conclusions, criteria-based decision making, and making use of relevant sources are considered.
\\\\
The research question is: "What is the feasibility of using reinforcement learning-based approaches for collision avoidance as opposed to computer vision and sensor-based approaches?" While the current implementation will not be enough to satisfy the requirements in the previous section, it does provide a lot of research products. It makes use of state-of-the-art techniques (curriculum/imitation learning) in a very relevant context (collision avoidance and autonomous drone control). Currently, the research will be continued by doing research into the feasibility of imitation learning. Based on this in combination with other relevant sources a conclusion will be made. This will then be followed by documenting the whole research, with properly formulating the research and sub questions, and describing the whole research process. The main improvement point is defining the research sub questions earlier. Due to the uncertain nature of the approach, this was postponed until the project took more shape. However, looking back making a preliminary list of sub questions would have been better.

\section{Professional Behavior}
Professional behavior aims at skills such as problem orientation, creativity, project management, productivity, self-reflection, and communicating within a professional environment.
\\\\
In order to maximize productivity, small quality improvements were introduced. First of all, as my attention span drops relatively fast, I prefer switching between small tasks a lot than to work on one long task for a long time. Thus, the daily planning has been designed so that there are always 2 different kinds of tasks available to avoid exhaustion from 1 type of task and thus reduced quality. Usually this would be a combination of reading/writing documentation and programming/designing. Working with this kind of system often allowed for additional online research time. Secondly, I consider structure an important thing. Because of this, aside from the long-term pivotal tracker planning made, a daily backlog with open to-dos has been made. While the company did have weekly stand up meetings, these were not very beneficial as most colleagues are not involved with my project. After communicating this with the supervisor, a private weekly progress meeting has been set up to show deliverables and notify him of important updates, as well as to ask questions.
\\\\
Improvement points here are finding a better balance between the graduation internship requirements and the company's wishes. An example of this being the scalability of the simulation. Also, I have difficulty determining what parts are up to the necessary standard when it comes to the activities, which sometimes leads to making things too complex. Starting simple and then building up should be the way to go.