\chapter{Planning}
\lhead{\thechapter \space Planning}
\label{ch:planning}

\begin{figure}[h]
	\centering
	\includegraphics[scale=0.65]{"img/screenshot_pivotaltracker"}
	\caption{Screenshot of Pivotal Tracker interface}
	\label{fig:screenshot_pivtracker}
\end{figure}

\noindent
Each of these epics have their own sub-activities, which are listed using a tool named Pivotal Tracker. Pivotal Tracker enables users to add stories with tasks to epics, which then can be assigned to people. A backlog is then generated. Pivotal Tracker will assist in fulfilling a requirement set by Seacon. Namely, the use of the Plan-Do-Check-Act cycle and agile development. A table containing the stories of each epic is given in appendix \ref{app:table}.\pagebreak

\begin{table}[h]
	\centering
	\begin{tabular}{|l|l|}
		\hline
		\textbf{Deliverable} & \textbf{Deadline} \\ \hline
		Project Plan & 27/09/2019 \\
		Midterm Report & 21/10/2019 \\
		Midterm Presentation & 28/10/2019 - 15/11/2019 \\
		Thesis Report & 13/01/2020 \\
		Final Presentation & 20/01/2020 - 07/02/2020 \\ \hline
	\end{tabular}
	\caption{List of school deadlines.}
	\label{tab:deadlines}
\end{table}

\noindent
The school also provided a set of deadlines for a set of reports and presentations. These reports and presentations are meant to document the start, middle, and end status of the project. The deadline dates can be found in table \ref{tab:deadlines}.