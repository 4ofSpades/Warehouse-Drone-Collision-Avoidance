\chapter{Scope}
\lhead{\thechapter \space Scope}
\label{ch:scope}

This chapter contains everything that should be explicitly mentioned regarding responsibilities, what will be in scope and what not, and information around the cost scope. \\\\
\noindent
This project will have a duration of one semester, or roughly 100 working days. The initiation is on 02/09/2019, and the final deliverable is due on 07/02/2020 latest. During this semester, the student will work the standard 40 hour workweek. Vacation days can be granted upon request, but do not add up to the working days.

\section{Student}
As stated in the \gls{FHTenL} regulations for graduation, during this project the student will not be burdened with tasks irrelevant to the assignment. The solution will thus exclusively consist out of software that at least is able to fulfill the minimal requirements stated in section \ref{sec:ass_desc}. The student will ultimately be held accountable for the completion of the project, on the condition that all other parties fulfill their responsibilities. \\\\
\noindent
Optional extensions are highlighted with a green outline in figure \ref{fig:roadmap}. These extensions include solutions that will allow actual drones to fly to a preset destination in a warehouse, as well as the inclusion of a barcode detector and scanner using the drone's camera with detection algorithms. 

\section{Seacon}
Seacon will be responsible for facilitating the student with a proper working environment and substantial supervision. Seacon will also be responsible for supplying an adequately equipped drone for testing purposes. This also includes handling the purchasing of additional equipment if the drone is deemed insufficiently equipped or dysfunctional by both the supervisor and the student. Additionally, Seacon will also be responsible for supplying the necessary hardware for computing-heavy tasks such as training a \gls{deep learning} model. Finally, Seacon should also be willing to assist in obtaining training material if necessary. Seacon has stated that the solution should be as cost efficient as possible. The student will be a compensation of \euro350 monthly.

\section{University}
\gls{FHTenL} will be responsible for assigning a university supervisor. This supervisor shall be tasked with providing feedback to i.e. reports and presentations of the student in order for the student to increase the quality of his end products and thesis. Another responsibility is giving advice regarding approaches, technologies, school-related requirements, and other contextual matters within certain limits. Finally, the student should be able to consult the university supervisor in case problems occur that might jeopardize the project.
