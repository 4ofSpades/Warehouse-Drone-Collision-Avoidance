\chapter{Risk Analysis}
\lhead{\thechapter \space Risk Management}
\label{ch:risks}

This chapter is dedicated to listing potential obstacles that might cause the project to stagnate. Especially in the context of \gls{AI} development, where there often is relied on trial-and-error, a lot of uncertainties present themselves. Analyzing potential risks helps with timely detecting them in order to find a workaround if necessary. \\\\
\noindent
In order to combat the majority of risks, there will be made use of agile development. weekly stand-up meetings will be held with the staff to discuss the progress and obstacles. From there on out, plans can be adjusted based on consultation from the company supervisor. The major risks are discussed in the next few sections.

\section{A.I. Risks}
A.I. development generally comes with a lot of risks. It is often hard to determine how a certain approach will work out other than trying it out. This could pose a problem when it comes to time management. However, the time lost could be minimized by doing proper research into approaches and alternatives, so that dead-ends can be recognized timely.

\section{Drone Risks}
As mentioned in chapter \ref{ch:scope}, the drone its cost should be kept to a minimum. This means that, initially, development should be done using only a front-facing camera-equipped drone. This might not be sufficient for a satisfactory solution, and while new equipment can be ordered, it will cost time nonetheless.

\section{Environmental Risks}
In the context of being a temporary employee, there are also general risks involved. Firstly, illness could pose a risk. Should this occur and become a problem, then this will be discussed with the student, company supervisor, and in severe cases the university coach. Secondly, misalignment of thoughts and miscommunication between the student and the company could distort results. Generally this should be solved internally, but in worst cases, the student will be consulted by the university coach. Finally, the risk of having an improperly defined scope is also present. This can come in forms of the assignment being too broad/specific, or the assignment being too big/small. This, however, could be averted through developing a proper project plan in combination with having regular progress meetings.

