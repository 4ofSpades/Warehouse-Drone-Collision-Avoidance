\chapter{Specific Requirements}
\lhead{\thechapter \space Specific Requirements}
\label{ch:specific_requirements}

\section{Functional Requirements}
The following section describes the functions the product shall, should, and might include. For the definitions of these 3 terms refer to the glossary. Each of these 3 terms form a section, which are then further segmented into status, maintenance, and action tasks, for which the definitions can also be found in the glossary. Each functional requirement will have the following format:
\begin{figure}[h]
	\centering
	\includegraphics[width=\linewidth]{img/func_req_format.png}
	\label{fig:func_req_format}
	\caption{Format for a functional requirement.}
\end{figure}

\noindent
Whereas:
\begin{itemize}
	\item \textbf{a} is the priority group (shall, should, might)
	\item \textbf{b} is the type of task
	\item \textbf{c} is the unique number of that requirement within its group.
\end{itemize}
Each requirement is prefixed by at least one set of brackets containing its unique ID, but may have a second set containing one or more IDs of prerequisite tasks or groups, each separated by comma. Note that in figure \ref{fig:func_req_format} the prerequisites bracket contains a single \textbf{"a"}, which indicates that all requirements in that priority group are a prerequisite to that task.\\\\

\textbf{[1] The product shall make the drone:}
\begin{itemize}
	\item \textbf{[1.1]} Status tasks:
	\begin{itemize}
		\item \textbf{[1.1.1]} Track the vertical center of the beam
		\item \textbf{[1.1.2]} Estimate the distance from the beam
		\item \textbf{[1.1.3]} Distinguish objects in the foreground from the background
		\item \textbf{[1.1.4]}[1.1.3, 1.2.1, 1.2.2] Check for objects in its trajectory
	\end{itemize}
	\item \textbf{[1.2]} Maintenance tasks:
	\begin{itemize}
		\item \textbf{[1.2.1]}[1.1.1] Maintain its altitude to keep the beam vertically centered
		\item \textbf{[1.2.2]}[1.1.2] Maintain a distance between 30 - 80 centimeters from the beam
	\end{itemize}
	\item \textbf{[1.3]} Action tasks:
	\begin{itemize}
		\item \textbf{[1.3.1]}[1.2] Move along the beam from left to right
		\item \textbf{[1.3.2]}[1.1.4] Avoid collisions by landing
	\end{itemize}
\end{itemize}
\noindent
\textbf{[2] The product should make the drone:}
\begin{itemize}
	\item \textbf{[2.1]} Status tasks:
	\begin{itemize}
		\item \textbf{[2.1.1]} Keep track of the remaining layers needing to be scanned
		\item \textbf{[2.1.2]} Determine the end of the beam
	\end{itemize}
	\item \textbf{[2.3]} Action tasks:
	\begin{itemize}
		\item \textbf{[2.3.1]}[1.1.2, 2.1] Move to other layers of a rack
		\item \textbf{[2.3.2]}[1.2] Move along the beam from right to left
		\item \textbf{[2.3.3]}[2.1.2] Stop at the end of the beam
	\end{itemize}
\end{itemize}
\noindent
\textbf{[3] The product might make the drone:}
\begin{itemize}
	\item \textbf{[3.1]} Status tasks:
	\begin{itemize}
		\item \textbf{[3.1.1]}[1.1.4] Determine if an object in its trajectory can be avoided
		\item \textbf{[3.1.2]} Determine pixel size:distance ratio of the beam
		\item \textbf{[3.1.3]}[1.2.1] Detect bar-codes on the beam
	\end{itemize}
	\item \textbf{[3.2]} Maintenance tasks:
	\begin{itemize}
		\item \textbf{[3.2.1]}[3.1.1] Readjust trajectory to avoid collision
	\end{itemize}
	\item \textbf{[3.3]} Action tasks:
	\begin{itemize}
		\item \textbf{[3.3.1]}[3.2.1] Move to the opposite rack of the same hallway
	\end{itemize}
	
\end{itemize}

\section{Assumptions}
In order for the product to function accordingly, the following assumptions have been made:
\begin{itemize}
	\itemsep0em
	\item The user is required to fly the drone to the correct starting position. This position is for the drone to face the left end of the lowest beam, at a distance between 30-80 centimeters.
	\item There are no things sticking out more than 10 centimeters from the rack.
	\item The drone will have enough battery capacity to complete one side of a rack.
	\item The warehouse interior lights are turned on and provide enough lighting.
	\item The drone is equipped with an RGB-camera.
	\item There is at least 3 meters of free space after the end of a beam.
	\item The drone remains connected to the device controlling it.
\end{itemize}