\chapter{Introduction}
\lhead{\thechapter \space Introduction}
\label{ch:introduction}

\section{Purpose}
%TO DO: Identify the product for which this document is written, as well as the intended audience
The purpose of this document is to provide a detailed description of the software product of the Drone Warehouse Collision Avoidance project. It contains the purpose and features of the software, the interfaces, and the constraints under which it must operate. This document is primarily intended for the department of Engineering \& IT of Seacon Logistics, as this department will contain/contains the future developers of the next iteration of this product. As this project is developed as a bachelor thesis project, the by Fontys University of Applied Sciences appointed tutors, lecturers, and experts are also considered an intended audience.

\section{Document Conventions \& Standards}
%TO DO: Describe any naming or typographical conventions, as well as the standards used (IEEE). For example, state whether priorities  for higher-level requirements are assumed to be inherited by detailed requirements, or whether every requirement statement is to have its own priority.
This document was created based on the IEEE Recommended Practice for Software Requirement Specifications standard 830-1998.

\section{Scope}
%TO DO: Summarize each of the named software products, including what they will and won't do. Also describe the application of the software, such as what the benefits, objectives, and or goals are.
The Warehouse Drone Collision Avoidance system (which is in this document generally referred to as "product") will primarily be responsible for providing a drone with a prototype solution to move along all layers of a single side of a rack in a warehouse, while avoiding collisions.

\section{Overview}
%TODO: List what the rest of this document will contain, and in what order.
The remainder of this document will feature an overall description of the product and the specific requirements. The former includes the perspective, information regarding a set of used interfaces, the product functions, and information regarding the product its users. The latter contains the specific requirements listed with unique identifiers, and a list of assumptions made in order for the product to function.

