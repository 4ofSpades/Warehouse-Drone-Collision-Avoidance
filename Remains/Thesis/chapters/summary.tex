\noindent
The warehouse drone collision avoidance project concerns itself with the analysis, design, and development of a solution to provide a drone with collision avoidance functionality. The time scope was set for 1 semester (5 months), and resulted in the creation of this bachelor thesis. This project contains 2 prototype concepts: A concept based on reinforcement learning and a concept based on combining computer vision with a state machine implementation. The reinforcement learning concept did not show promising results as it did not show any sign of learning. The state machine concept its components work separately, but still requires smoother transitioning and a number of bugs needed to be fixed before it is able to run the entire state machine properly. This project also contained a research aspect. This research focused on the viability of Artificial Intelligence techniques for collision avoidance. Out of this came 3 approaches, of which one was created a prototype for. From the research it can be concluded that a single, stand-alone reinforcement learning algorithm is insufficient when attempting to use drones for cycle counting.