\chapter{Conclusion}
\lhead{\thechapter \space Conclusion}
\label{ch:conclusion}
The Warehouse Drone Collision Avoidance project resulted in the development of 2 concepts: an \gls{AI}-based one using Unity's ML-Agents and a state machine implementation one. This chapter gives a short summary of each of the concepts and draws a conclusion.

\section{Machine Learning Agents \& Research}
For the \gls{mlagents} concept a Unity environment was developed that could simulate a warehouse with variations. Using the \gls{mlagents} toolkit several training sessions were run. This concept, however, did not show any signs of learning after having trained it for 3 hours for \gls{RL} and 1 hour for \gls{IL}. After having done more research, it can be concluded that reinforcement learning as stand-alone solution will not suffice. This is due to the fact that rewards are very sparsely distributed, which is generally handled badly by \gls{RL} algorithms. Moreover, as environments get increasingly complex, the risk of the algorithm getting stuck in local optima or finding an out-of-the-box answer that gives a better reward than the intended answer increases \citep{rlblogpost}. An example is the paper by \gls{AI} research company OpenAI on a game where 4 agents are tasked to play hide and seek in teams of 2. In the paper, it is described that eventually the agents started finding several exploits in the physics engine of the game, with an example being abusing the contact physics to force a ramp to go through the wall separating the play area \citep{hideandseek}.
\\\\
Based on the results and the risk of not having selected a viable approach, it was decided to switch to the state machine concept after the midterm presentation. This was done as a (partially) working product was favored over continuing the research.

\section{State Machine Concept}
Unlike the \gls{mlagents} approach, the state machine concept limited its scope to just one side of a rack. Through the use of a state machine, the product is able to switch tasks based on a set of criteria. Currently included tasks are: finding the vertical distance between the center of the camera and the center of the beam, estimating the distance between the drone and the beam, moving sideways, rotate, distinguish the foreground from the background, and make a rough estimation of how close an object is. Although the product contains an implementation for all these tasks, it is concluded that it is not ready for a practical environment yet. Tests using the Tello drone show that the product still has insufficient control over the drone, and is still too susceptible to noise. Furthermore, the drone still contains some bugs and improvement points, such as the rotation bug mentioned in section \ref{sec:state_machine_limits} and the fact that the collision detector its performance is based on well the drone can maintain its position. With some optimization, however,  it is believed that this concept could end up being usable in a practical environment such as a warehouse.
