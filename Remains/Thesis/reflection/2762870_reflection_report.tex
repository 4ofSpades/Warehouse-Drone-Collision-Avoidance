%%%%%%%%%%%%%%%%%%%%%%%%%%%%%%%%%%%%%%%%%%%%%%%%%%
% PAGE LAYOUT
%%%%%%%%%%%%%%%%%%%%%%%%%%%%%%%%%%%%%%%%%%%%%%%%%%
\documentclass[a4paper,11pt]{report}
%\usepackage[subtle]{savetrees}
\usepackage{fullpage}
\usepackage{wrapfig}
%%%%%%%%%%%%%%%%%%%%%%%%%%%%%%%%%%%%%%%%%%%%%%%%%%
%% PAGEHEADER STYLING
%%%%%%%%%%%%%%%%%%%%%%%%%%%%%%%%%%%%%%%%%%%%%%%%%%
\usepackage{etoolbox,fancyhdr,xcolor}
%%%%%%%%%%%%%%%%%%%%%%%%%%%%%%%%%%%%%%%%%%%%%%%%%%
% EXTERNAL INCLUDES
%%%%%%%%%%%%%%%%%%%%%%%%%%%%%%%%%%%%%%%%%%%%%%%%%%
\usepackage{graphicx}
\usepackage{svg}
\usepackage{csvsimple}
\usepackage{pdfpages}
\usepackage{hyperref}
\usepackage{lscape}
\usepackage{eurosym}
%%%%%%%%%%%%%%%%%%%%%%%%%%%%%%%%%%%%%%%%%%%%%%%%%%
%% BIBLIOGRAPHY SETTINGS
%%%%%%%%%%%%%%%%%%%%%%%%%%%%%%%%%%%%%%%%%%%%%%%%%%
\usepackage[comma]{natbib}
\usepackage{color, colortbl}
\usepackage[nonumberlist,acronym]{glossaries}
%%%%%%%%%%%%%%%%%%%%%%%%%%%%%%%%%%%%%%%%%%%%%%%%%%
% CAPTIONS AND REFERENCING
%%%%%%%%%%%%%%%%%%%%%%%%%%%%%%%%%%%%%%%%%%%%%%%%%%
\usepackage{fancyref}
%\usepackage{image_captioning}
%%%%%%%%%%%%%%%%%%%%%%%%%%%%%%%%%%%%%%%%%%%%%%%%%%
%% CODE SNIPPET LISTING SETTINGS
%%%%%%%%%%%%%%%%%%%%%%%%%%%%%%%%%%%%%%%%%%%%%%%%%%
\usepackage{listings}
\renewcommand\lstlistlistingname{List of Listings}
\lstset{numbers=left,xleftmargin=2em,captionpos=b}
%%%%%%%%%%%%%%%%%%%%%%%%%%%%%%%%%%%%%%%%%%%%%%%%%%
%    SECTIONS
%%%%%%%%%%%%%%%%%%%%%%%%%%%%%%%%%%%%%%%%%%%%%%%%%%
\usepackage{titlesec}
\usepackage{sectsty}
\usepackage{csquotes}
%%%%%%%%%%%%%%%%%%%%%%%%%%%%%%%%%%%%%%%%%%%%%%%%%%
%% APPENDIX SETTINGS
%%%%%%%%%%%%%%%%%%%%%%%%%%%%%%%%%%%%%%%%%%%%%%%%%%
\usepackage[titletoc]{appendix}
%\renewcommand\appendixtocname{Appendices}
%\renewcommand\appendixpagename{Appendices}
%All non-canonical parts/chapters should be numbered with roman numbers
\pagenumbering{roman}
\pdfminorversion=7


% Title Page
\title{Self-Reflection Report}
\subtitle{Warehouse Drone Collision Avoidance}
\author{Tristan van Vegchel}

\date{Venlo, \today}
\requirement{the degree \\ 
	Bachelor of Science in Informatics \\
	To be awarded by the \\
	Fontys Hogeschool Techniek en Logistiek}


\begin{document}
\maketitle
%%%%%%%%%%%%%%%%%%%%%%%%%%%%%%%%%%%%%%%%%%%%%%%%%%
%% CONFIGURATION OF PAGE HEADER
%%%%%%%%%%%%%%%%%%%%%%%%%%%%%%%%%%%%%%%%%%%%%%%%%%
\pagestyle{fancy}
\newcommand{\headrulecolor}[1]{\patchcmd{\headrule}{\hrule}{\color{#1}\hrule}{}{}}
\newcommand{\footrulecolor}[1]{\patchcmd{\footrule}{\hrule}{\color{#1}\hrule}{}{}}
\pagestyle{fancy}
\fancyhf{}% Clear header/footer
\fancyhead[C]{}
\fancyhead[R]{\thepage}
\setlength{\headsep}{33pt}
\setlength{\headheight}{13.6pt}
\renewcommand{\headrulewidth}{0.4pt}
\setlength{\parindent}{0em}
\pagenumbering{arabic}

\noindent
This report covers a self-reflection. It covers 3 subjects: the research done, the product developed, and professional behavior.
\\\\
Let's start things off with the research done. First of all, I think that the research fitted in well with the project. Seacon has been very interested in the application of artificial intelligence, and this project is a good first case for it. Especially as drones are rapidly decreasing in price and companies are becoming increasingly aware of it, the demand for intelligent drones is likely to drastically increase over time. That said, the scale for the scope of this project was way too big. There is an abundance of Artificial Intelligence (A.I.) techniques, after all. The main research question should have been limited to just genetic algorithms or even just reinforcement learning. Aside from that I also think that, especially pre-midterm, the approaches weren't well documented enough. While I did make use of several sources, I did not document it properly. Also, I did not make a list of sub questions for the research explicitly, which would have clarified the requirements better.
\\\\
Secondly, I would like to address the product. 2 concepts were made: one using reinforcement learning, and one using a state machine implementation. While I think both contain a lot of valuable information, it is a shame they have so little overlap. It's almost like I did 2 separate projects. Also, I decided to take on this task because of the A.I. part primarily, so it was rather saddening to have to halt the research in favor of the certainty of delivering a partially working product. As for the individual implementations, I am quite satisfied with the effort. While the analysis should have been more explicitly documented, I paid careful attention to the designs and implementations of the concepts with regard to scalability, maintainability, and usage. An example of this is that all the code is documented in detail, or code being structured using the single responsibility as much as possible.
\\\\
Finally, there's the professional behavior part. This covers the more general stuff of working in a professional environment. In all honesty, I did not really get the feeling I was operating on a professional level that much. In my defence, however, I believe there was not really all that much room for it. I was tasked to do this project alone, which made sitting in an office feel a little superficial as there was almost no interaction with other people. Furthermore, I couldn't really take advantage of things like code reviews that much, as there are no Python/A.I. developers. In order to keep myself on the right track, however, I asked my supervisor for weekly progress meetings. During these meetings the things that were done that week were discussed, documents were reviewed, and advise on the analysis and design was sometimes given. Whenever I got stuck or lacked information I quickly made use of the opportunity to ask, though.
\\\\
Overall, I do believe there is great potential in both concepts I developed, as well as the research I've done. I hope to be able to extend it one day, but preferably in a small team. The biggest mistake I made during this project is not defining what is supposed to be done properly enough, which costed me a lot of time. Because of this loss, I ended up with a product of which I am not completely satisfied with.

\end{document}          
